\begin{enumerate}
  \item \textbf{Tense}: In general and to make it simple:
  \begin{itemize}
    \item Use the simple past tense in experiments (what you did and what the model did) and related work (what others did); Use the simple present tense for discussions, model descriptions, known facts, etc.
    \item If you're not sure, keep it simple (the simple present tense).
  \end{itemize}

  \item \textbf{Abbreviation}:
  \begin{itemize}
    \item
      For an abbreviation, introduce it at its first occurrence in the paper. Be careful about plurals.
      See the example in Sec~\ref{sec:spacing}.
  \end{itemize}

  \item \textbf{Oxford Comma}\footnote[3]{See \url{https://en.wikipedia.org/wiki/Serial\_comma}.}: When listing $\ge 2$ items, use the Oxford comma (a comma before \textit{and} and \textit{or}) for clarity.

\end{enumerate} 
