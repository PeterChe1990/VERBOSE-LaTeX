
A few examples: (You may skip these examples.)

\begin{itemize}
  \item A (quite common) double-column table.
  \begin{quote}\begin{scriptsize}\begin{verbatim}
\begin{table*}[t]
  \centering
  \caption{A double-column table with several useful modules (multiple rows/columns, different lines, etc.)}
  %\resizebox{0.99\linewidth}{!}{
    \begin{tabular}{cl|cc|cc|ccc}
      \toprule
      \multicolumn{2}{c|}{\multirow{2}{*}{Multi-Row \& Multi-Column}} & \multicolumn{2}{c|}{Setting-1}
          & \multicolumn{2}{c|}{Setting-2} &  \multicolumn{3}{c}{Setting-3} \\
      & & Col-1-1 & Col-1-2 & Col-2-1 & Col-2-2 & Col-3-1 & Col-3-2 & Col-3-3 \\
      \midrule
      \multirow{4.5}{*}{Dataset-1} & Method~1 & 1.234 & 1.234 & 1.234 & 1.234 & 1.234 & 1.234 & 1.234 \\
      & Method~2 & 1.234 & 1.234 & -- & -- & 1.234 & 1.234 & 1.234 \\
      & The Strongest Baseline & 7.890 & 7.890 & 7.890 & 7.890 & 7.890 & 7.890 & 7.890 \\
      \cmidrule(lr){2-9}
      & \textbf{Ours} & \textbf{9.876} & \textbf{9.876} & \textbf{9.876} & \textbf{9.876}
          & \textbf{9.876} & \textbf{9.876} & \textbf{9.876} \\
      \cmidrule{1-9}
      \multirow{2.5}{*}{Dataset-2} & A Weak Baseline & 1.234 & 1.234 & 1.234 & 1.234
          & 1.234 & 1.234 & 1.234 \\
      \cmidrule(lr){2-9}
      & \textbf{Ours} & \textbf{9.876} & \textbf{9.876} & \textbf{9.876} & \textbf{9.876}
          & \textbf{9.876} & \textbf{9.876} & \textbf{9.876} \\
      \bottomrule
    \end{tabular}
  %}
  \label{tab:double}
\end{table*}
  \end{verbatim}\end{scriptsize}\end{quote}

  \item A single-column table.
  \begin{quote}\begin{scriptsize}\begin{verbatim}
\begin{table}[h]
  \centering
  \caption{A single-column table with more components.}
  \resizebox{0.97\columnwidth}{!}{
    \begin{tabular}{ccc|c|l|r}
      \toprule
      S1 & S2 & S3 & Value & Comment & Another Column \\
      \midrule
      -- & -- & $\checkmark$ & 1.23E4 & This is a comment. & A comment. \\
      \cmidrule(lr){1-6}
      -- & $\checkmark$ & $\checkmark$ & 1.23E4 & \makecell[cl]{This is a longer \\ text with two lines.}
          & Right-aligned texts. \\
      \bottomrule
    \end{tabular}
  }
  \label{tab:single}
\end{table}
  \end{verbatim}\end{scriptsize}\end{quote}

\end{itemize}

\twocolumn

\begin{table*}[t]
  \centering
  \caption{A double-column table with several useful modules (multiple rows/columns, different lines, etc.)}
  %\resizebox{0.99\linewidth}{!}{
    \begin{tabular}{cl|cc|cc|ccc}
      \toprule
      \multicolumn{2}{c|}{\multirow{2}{*}{Multi-Row \& Multi-Column}} & \multicolumn{2}{c|}{Setting-1} & \multicolumn{2}{c|}{Setting-2} &  \multicolumn{3}{c}{Setting-3} \\
      & & Col-1-1 & Col-1-2 & Col-2-1 & Col-2-2 & Col-3-1 & Col-3-2 & Col-3-3 \\
      \midrule
      \multirow{4.5}{*}{Dataset-1} & Method~1 & 1.234 & 1.234 & 1.234 & 1.234 & 1.234 & 1.234 & 1.234 \\
      & Method~2 & 1.234 & 1.234 & -- & -- & 1.234 & 1.234 & 1.234 \\
      & The Strongest Baseline & 7.890 & 7.890 & 7.890 & 7.890 & 7.890 & 7.890 & 7.890 \\
      \cmidrule(lr){2-9}
      & \textbf{Ours} & \textbf{9.876} & \textbf{9.876} & \textbf{9.876} & \textbf{9.876} & \textbf{9.876} & \textbf{9.876} & \textbf{9.876} \\
      \cmidrule{1-9}
      \multirow{2.5}{*}{Dataset-2} & A Weak Baseline & 1.234 & 1.234 & 1.234 & 1.234 & 1.234 & 1.234 & 1.234 \\
      \cmidrule(lr){2-9}
      & \textbf{Ours} & \textbf{9.876} & \textbf{9.876} & \textbf{9.876} & \textbf{9.876} & \textbf{9.876} & \textbf{9.876} & \textbf{9.876} \\
      \bottomrule
    \end{tabular}
  %}
  \label{tab:double}
\end{table*}


\begin{table}[h]
  \centering
  \caption{A single-column table with more components.}
  \resizebox{0.97\columnwidth}{!}{
    \begin{tabular}{ccc|c|l|r}
      \toprule
      S1 & S2 & S3 & Value & Comment & Another Column \\
      \midrule
      -- & -- & $\checkmark$ & 1.23E4 & This is a comment. & A comment. \\
      \cmidrule(lr){1-6}
      -- & $\checkmark$ & $\checkmark$ & 1.23E4 & \makecell[cl]{This is a longer \\ text with two lines.} & Right-aligned texts. \\
      \bottomrule
    \end{tabular}
  }
  \label{tab:single}
\end{table}



\begin{enumerate}
  \item \textbf{Separated Files}:
    Put each table or a group of consecutive tables (that are expected to be placed together) into a separate \texttt{.tex} file.
    Name the tex file as \texttt{tab-xxx.tex}, or put all these tex files into a folder named \texttt{./table/}.

  \item \textbf{One/Two-Column}:
    For a two-column PDF, use \texttt{table} to create a single-column figure~(Table~\ref{tab:single}), and use \texttt{table*} to create a double-column table~(Table~\ref{tab:double}).
    For a single-column PDF, use the two environments interchangeably.

  \item \textbf{Position}:
    Same as figures.

  \item \textbf{Centering}:
    Same as figures.

  \item \textbf{Table Size}:
  \begin{itemize}
    \item
      Use \texttt{\char`\\resizebox\{xx\}\{!\}\{...\}}, to set the table width.
      Here, \texttt{xx} can be a fraction of \texttt{\char`\\columnwidth} (for single-column tables), \texttt{\char`\\textwidth}, and \texttt{\char`\\linewidth} (for double-column figures).
    \item
      Following common practice, adjust the table size to make the table font size \textit{slightly smaller} than the main text.
      Not too small nor too large.
    \item
      Always keep the \texttt{\char`\\resizebox} lines in your TeX file; If you do not want to adjust the table size, simply comment it out rather than deleting it.
      
    \item To make tables in multiple pages, use the package \texttt{longtable}.
    \begin{itemize}
      \item  Use \texttt{\char`\\\char`\\\char`\*} to enforce (mid/cmid/top/bottom-) rules not being pushed to the next page.
    \end{itemize}

  \end{itemize}

  \item \textbf{Captions \& Labels}:
  \begin{itemize}
    \item The captions/subcaptions can either be a phrase, a sentences, or multiple sentences. A sentence-ending period is needed.
    \item The captions are usually above the tables. Some venues require captions being below the tables. Please stick to the venues' provided templates.
    \item If any texts or figure parts are not described, you should describe them in the caption.
    \item For caption labels, refer to Section~\ref{sec:references}.
  \end{itemize}

  \item \textbf{Alignment}:
    For each column, use \texttt{l/c/r} to left/center/right-align the contents.

  \item \textbf{Multiple Columns/Rows}:
  \begin{itemize}
    \item
      \texttt{\char`\\multicolumn\{x\}\{y\}\{...\}}: For cells with $>1$ columns.
      The parameters are:
    \begin{itemize}
      \item \texttt{x}: \# of columns;
      \item \texttt{y}: alignment (\texttt{l/c/r}) and vertical lines.
    \end{itemize}
    \item
      \texttt{\char`\\multirow\{x\}\{y\}\{...\}}: For cells with $>1$ rows.
      The parameters are:
    \begin{itemize}
      \item \texttt{x}: \# of rows (can be non-integer);
      \item \texttt{y}: width (by default `\texttt{*}').
    \end{itemize}
    \item For cells with both $>1$ columns and $>1$ rows: Use nested \texttt{\char`\\multicolumn} and \texttt{\char`\\multirow}.
    \item
      \texttt{\char`\\makecell[xy]\{...\}}: For multilined cells.
      The parameters are:
    \begin{itemize}
      \item \texttt{x}: vertical alignment (\texttt{t/c/b});
      \item \texttt{y}: horizontal alignment (\texttt{l/c/r}).
    \end{itemize}
  \end{itemize}

  \item \textbf{Horizontal Lines (Rules)}: For examples, see Table~\ref{tab:double}.
  \begin{itemize}
    \item
      Use rules from the \texttt{booktabs} package rather than lines: \texttt{\char`\\toprule}, \texttt{\char`\\midrule}, \texttt{\char`\\bottomrule}, and \texttt{\char`\\cmidrule}.
    \item
      Put one and only one \texttt{\char`\\toprule}/\texttt{\char`\\bottomrule} at the very top/bottom of the table.
      They are thicker solid lines and properly aligned.
    \item
      Use one \texttt{\char`\\midrule} to separate the header and the body.
      It is thinner than the top/bottom rules.
    \item
      Use \texttt{\char`\\cmidrule\{x-y\}} to separate different settings (blocks of lines).
      This line starts from \texttt{x}-th column and ends at \texttt{y}-th column.
    \item
      Use \texttt{\char`\\cmidrule(lr)\{x-y\}} to separate our method and the others.
      Only used if necessary.
      \texttt{lr} means the line trims in the left/right.

  \end{itemize}

  \item \textbf{Vertical Lines}:
    `\texttt{|}'. Use it seldomly.
    \texttt{\char`\\multicolumn} overrides the table-level vertical lines for the spanned columns, so you need to add the vertical line argument in it. See Table~\ref{tab:double} as an example.

  \item \textbf{Texts and Numbers}:
  \begin{itemize}
    \item Make sure numbers (for the same metric) have the same number of decimal places.
    \item
      Put numbers in text mode, and use \texttt{\char`\\textbf} and \texttt{\char`\\textit} to emphasize.
      Otherwise, if you would like to put number is math mode (within \texttt{\$...\$}), use \texttt{\char`\\mathbf} and  \texttt{\char`\\mathit} to emphasize.
      Make sure your choice is consistent. The former one is preferred.
  \end{itemize}

\end{enumerate}


