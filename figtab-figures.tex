A few examples: (You may skip these examples.)

\begin{itemize}
  \item The most common single-column figure.
    \begin{quote}\begin{scriptsize}\begin{verbatim}
\begin{figure}[h!]
  \centering
  \includegraphics[width=.9\columnwidth]{fig/fig1.pdf}
  \caption{An example of a single-column figure.}
  \label{fig:single}
\end{figure}
    \end{verbatim}\end{scriptsize}\end{quote}

  \item A two-column figure.
    \begin{quote}\begin{scriptsize}\begin{verbatim}
\begin{figure*}[t]
  \centering
  \includegraphics[width=.95\linewidth]{fig/fig1.pdf}
  \caption{An example of a two-column figure at top.}
  \label{fig:double}
\end{figure*}
    \end{verbatim}\end{scriptsize}\end{quote}

  \item Multiple figures in a row (in two ways).
    \begin{quote}\begin{scriptsize}\begin{verbatim}
\begin{figure}[h!]
  \centering
  \begin{subfigure}[b]{0.39\columnwidth}
    \centering
    \includegraphics[width=.3\textwidth]{fig/fig1.pdf}
    \caption{Left.}
    \label{fig:subcaption-two-inner-left}
  \end{subfigure}
  \begin{subfigure}[b]{0.59\columnwidth}
    \centering
    \includegraphics[width=.5\textwidth]{fig/fig1.pdf}
    \caption{Right.}
    \label{fig:subcaption-two-inner-right}
  \end{subfigure}
  \caption{The outer caption of the two figures.}
  \label{fig:subcaption}
\end{figure}

\begin{figure}[h!]
  \centering
  \includegraphics[width=.12\columnwidth]{fig/fig1.pdf}
  \hspace{.1in}
  \includegraphics[width=.3\columnwidth]{fig/fig1.pdf}
  \hspace{.1in}
  \includegraphics[width=.12\columnwidth]{fig/fig1.pdf}
  \caption{Multiple figures without subcaption.}
  \label{fig:multiple}
\end{figure}
    \end{verbatim}\end{scriptsize}\end{quote}

  \item Figures in a $4 \times 2$ grid.
    \begin{quote}\begin{scriptsize}\begin{verbatim}
\begin{figure}[h!]
  \centering
  \begin{subfigure}[b]{0.24\columnwidth}
    \centering
    \includegraphics[width=1\textwidth]{fig/fig1.pdf}\\
    \includegraphics[width=1\textwidth]{fig/fig1.pdf}
    \caption{LM.}
    \label{fig:grid-left-most}
  \end{subfigure}
  \begin{subfigure}[b]{0.24\columnwidth}
    \centering
    \includegraphics[width=1\textwidth]{fig/fig1.pdf}\\
    \includegraphics[width=1\textwidth]{fig/fig1.pdf}
    \caption{L.}
    \label{fig:grid-middle-left}
  \end{subfigure}
  \begin{subfigure}[b]{0.24\columnwidth}
    \centering
    \includegraphics[width=1\textwidth]{fig/fig1.pdf}\\
    \includegraphics[width=1\textwidth]{fig/fig1.pdf}
    \caption{R.}
    \label{fig:grid-middle-right}
  \end{subfigure}
  \begin{subfigure}[b]{0.24\columnwidth}
    \centering
    \includegraphics[width=1\textwidth]{fig/fig1.pdf}\\
    \includegraphics[width=1\textwidth]{fig/fig1.pdf}
    \caption{RM.}
    \label{fig:grid-right-most}
  \end{subfigure}
  \caption{Multiple figures in grid.}
  \label{fig:grid}
\end{figure}
    \end{verbatim}\end{scriptsize}\end{quote}

\end{itemize}

\begin{figure}[h!]
  \centering
  \includegraphics[width=.9\columnwidth]{fig/fig1.pdf}
  \caption{An example of a single-column figure.}
  \label{fig:single}
\end{figure}

\begin{figure*}[t]
  \centering
  \includegraphics[width=.95\linewidth]{fig/fig1.pdf}
  \caption{An example of a two-column figure at top.}
  \label{fig:double}
\end{figure*}

\begin{figure}[h!]
  \centering
  \begin{subfigure}[b]{0.39\columnwidth}
    \centering
    \includegraphics[width=.3\textwidth]{fig/fig1.pdf}
    \caption{Left.}
    \label{fig:subcaption-two-inner-left}
  \end{subfigure}
  \begin{subfigure}[b]{0.59\columnwidth}
    \centering
    \includegraphics[width=.5\textwidth]{fig/fig1.pdf}
    \caption{Right.}
    \label{fig:subcaption-two-inner-right}
  \end{subfigure}
  \caption{The outer caption of the two figures.}
  \label{fig:subcaption}
\end{figure}

\begin{figure}[h!]
  \centering
  \includegraphics[width=.12\columnwidth]{fig/fig1.pdf}
  \hspace{.1in}
  \includegraphics[width=.3\columnwidth]{fig/fig1.pdf}
  \hspace{.1in}
  \includegraphics[width=.12\columnwidth]{fig/fig1.pdf}
  \caption{Multiple figures without subcaption.}
  \label{fig:multiple}
\end{figure}

\begin{figure}[h!]
  \centering
  \begin{subfigure}[b]{0.24\columnwidth}
    \centering
    \includegraphics[width=1\textwidth]{fig/fig1.pdf}\\
    \includegraphics[width=1\textwidth]{fig/fig1.pdf}
    \caption{LM.}
    \label{fig:grid-left-most}
  \end{subfigure}
  \begin{subfigure}[b]{0.24\columnwidth}
    \centering
    \includegraphics[width=1\textwidth]{fig/fig1.pdf}\\
    \includegraphics[width=1\textwidth]{fig/fig1.pdf}
    \caption{L.}
    \label{fig:grid-middle-left}
  \end{subfigure}
  \begin{subfigure}[b]{0.24\columnwidth}
    \centering
    \includegraphics[width=1\textwidth]{fig/fig1.pdf}\\
    \includegraphics[width=1\textwidth]{fig/fig1.pdf}
    \caption{R.}
    \label{fig:grid-middle-right}
  \end{subfigure}
  \begin{subfigure}[b]{0.24\columnwidth}
    \centering
    \includegraphics[width=1\textwidth]{fig/fig1.pdf}\\
    \includegraphics[width=1\textwidth]{fig/fig1.pdf}
    \caption{RM.}
    \label{fig:grid-right-most}
  \end{subfigure}
  \caption{Multiple figures in grid.}
  \label{fig:grid}
\end{figure}



\begin{enumerate}
  \item \textbf{Separated Files}:
    Put each figure or a group of consecutive figures (that are expected to be placed together) into a separate \texttt{.tex} file.
    \footnote{Moving one line (\texttt{\char`\\input\{xxx.tex\}}) around is much easier and cleaner than moving a long scripts.}
    It is suggested to name the tex file as \texttt{fig-xxx.tex} where \texttt{xxx} is the same as the figure file name, or put all these tex files into a folder named \texttt{./figure/}.

  \item Before you put the figures in tex files:
    \begin{itemize}
      \item \textbf{File Type}:
        Export your figures as \textbf{PDF} files with \textit{font embedded}. No \textit{.png} nor \textit{.jpg}.
        Some journals require \textit{.eps} for the camera ready, and you can convert PDF files into EPS files by the \texttt{pdftops} command.
  
      \item Please \textbf{embed the fonts} in the figures. Serif typefaces are preferred. Double check the contents to be properly rescaled, capitalized, and with no typos.
  
      \item Do \textbf{NOT} use \texttt{crop} in {\LaTeX}. You can use the bash command \texttt{pdfcrop} (Installed with TeX) to remove margins or edit figures in Painter/PowerPoint.
    \end{itemize}

  \item \textbf{One/Two-Column}:
    For a two-column PDF, use \texttt{figure} to create a single-column figure~(Figure~\ref{fig:single}), and use \texttt{figure*} to create a double-column figure~(Figure~\ref{fig:double}).
    For a single-column PDF, use the two environments interchangeably.

  \item \textbf{Position}:
    Use \texttt{h/t/b} to place the figure here/at page top/at page bottom. `\texttt{!}' enforces your option.

    \begin{itemize}
      \item Notice how {\LaTeX} places figures: It tries to firstly settle down all TeX codes before this figure.
        Then, it tries to fit this figure with the position command (\texttt{h/t/b}).
        If the figure can not be put on this page, it will be put on the next page \textbf{and} after/below all previous figures.
        Therefore, a figure often shows later than where you want; And you need to manually move ahead this figure's TeX code.
    \end{itemize}

  \item \textbf{Centering}:
    Always center your figures. Use \texttt{\char`\\centering} instead of \texttt{\char`\\begin\{center\}}.\
    \footnote{The latter one brings undesired extra vertical spaces.}

  \item \textbf{Figure Size}:
    Use \texttt{[width=xx]} to set the figure width.
    Here, \texttt{xx} can be a fraction of \texttt{\char`\\columnwidth} (for single-column figures), \texttt{\char`\\textwidth}, and \texttt{\char`\\linewidth} (for double-column figures).

  \item \textbf{Captions \& Labels}:
    \begin{itemize}
      \item The captions/subcaptions can either be a phrase, a sentences, or multiple sentences. A sentence-ending period is needed.
      \item The captions are always under the figures.
      \item If any texts or figure parts are not described, you should describe them in the caption.
      \item For caption labels, refer to Section~\ref{sec:references}.
    \end{itemize}

  \item \textbf{Subplots}:
    \begin{itemize}
      \item Use package \texttt{subcaption} if you want to use captions for subplots~(Figures~\ref{fig:subcaption} and~\ref{fig:grid}); For subplots without captions, simply use multiple \texttt{\char`\\includegraphics}~(Figures~\ref{fig:multiple} and~\ref{fig:grid}).
      \item Make sure the summed up width is less than 1 columnwidth/linewidth for single/double-column figures; Otherwise, the figures won't be on a single line.
      \item You may adjust the spaces between figures by inserting \texttt{\char`\\hspace\{\}}. Please put them evenly.
    \end{itemize}

  \item \textbf{Warning}:
    \begin{itemize}
      \item To suppress warnings on multiple page groups: include \texttt{\char`\\pdfsuppresswarningpagegroup=1}.
    \end{itemize}
\end{enumerate}
