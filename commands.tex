Use a single file (e.g., \texttt{macro.tex}) for all user-defined common commands. Keep and expand this file towards your Swiss Army Knife.
Here I summarize a few useful tips.

\textbf{For full details please refer to} \texttt{macro.tex}.

\begin{itemize}
  \item Put \texttt{\char`\\input\{macro.tex\}} in the main tex file, after all packages and commands provided by conference/journal template and before \texttt{\char`\\begin\{document\}}.
  \item What we should put in the \texttt{macro.tex} file (in the following orders):
  \begin{enumerate}
    \item \textit{Packages} to be used.
    \item \textit{General commands} for all papers. E.g., math, format, editing, and rebuttal. They're rarely modified.
    \item \textit{Special commands} for this paper only: definitions of terms/abbrivations, shortcuts for frequently-used math variables, hacks of length/margin settings, etc.
  \end{enumerate}

  \item \textbf{Packages}:
  \begin{itemize}
    \item To use new packages:
    \begin{quote}\begin{scriptsize}\begin{verbatim}
\usepackage[OPTIONS]{PACKAGENAME}
    \end{verbatim}\end{scriptsize}\end{quote}

    \item Make sure the used packages are allowed by the conference/journal template.
  \end{itemize}

  \item \textbf{Commands}:
  \begin{itemize}
    \item To define new commands:
    \begin{quote}\begin{scriptsize}\begin{verbatim}
\newcommand{\COMMANDNAME}[NUMBER]{DEFINITION}
\renewcommand{\COMMANDNAME}[NUMBER]{DEFINITION}
\def\COMMANDNAME{DEFINITION}    % Less used.
    \end{verbatim}\end{scriptsize}\end{quote}

    \begin{itemize}
      \item Use \texttt{\char`\\newcommand} for new commands, and \texttt{\char`\\newcommand} for overriding existing commands.
      \item \texttt{NUMBER} is the number of parameters for the command (default value: 0).
      \item Use `\#$x$' for $x$-th parameter in the definition.

      \item
        Use new commands for model names, module names, etc.
        Thus, you only need to change it once in the command definition, without seeking it at all places (slow and error-prone).
        See the example below.
      \begin{quote}\begin{scriptsize}
        \begin{verbatim}
We propose a novel model, named {\ourslong}~(\ours),
which, however, sounds boring and performs poorly.
        \end{verbatim}

        We propose a novel model, named {\ourslong}~(\ours), which, however, sounds boring and performs poorly.
      \end{scriptsize}\end{quote}

      \item 
        When using the command in texts, a pair of bracket is needed, so that a proper space is inserted. 
        Otherwise, \texttt{\char`\\xspace} can be used.
        See the examples below.

      \begin{quote}\begin{scriptsize}
        \begin{verbatim}
The use of \ours is wrong. That of {\ours} is right.
The use of \oursx is easy but risky.
        \end{verbatim}
        The use of \ours is wrong. That of {\ours} is right.
        The use of \oursx is easy but risky.
      \end{scriptsize}\end{quote}

      \end{itemize}
    \end{itemize}

  \item
    Commands for editing and commenting:
    E.g., \texttt{\char`\\todo} for \todo{mild reminders} and \texttt{\char`\\TODO} for \TODO{Critical Issues}.
    Also, define personal comment command for each collaborator (with different colors), e.g., \texttt{\char`\\zc} \zc{for me}.

  \item Command for the 3rd level headline:
    While \texttt{\char`\\section} and \texttt{\char`\\subsection} are used for the top-2 level headlines,
    three variations of \texttt{\char`\\para} are provided for the 3rd level headline.
    You can switch among them to get different (wide/compact) spacing. To see the differences:

  {
    \hrulefill

      \subsubsection{Caption by Subsubsection} This is formal and takes a lot of space.

      \paragraph{Caption by Paragraph} This is commonly used.

      \noindent\textbf{Caption by Textbf.\ } This is the most compact one.

    \hrulefill
  }

  \item Commands for rebuttal: Some conferences require a single-page PDF for rebuttal; Use \texttt{\char`\\question} to refer to the reviewer's question.
  \begin{quote}\begin{scriptsize}
    \begin{verbatim}
\question{R1Q2}{important baseline xxx} We
compared it, and new results are ...
    \end{verbatim}
    \question{R1Q2}{important baseline xxx} We compared it, and new results are ...
  \end{scriptsize}\end{quote}

  \item Commands for revision: use \texttt{\char`\\dif} to mark the updates differences/updates of the revision compared to the original manuscript. Useful for journal resubmissions.

\end{itemize} 