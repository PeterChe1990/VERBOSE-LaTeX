%%%% %%%% %%%% %%%% \input{macro.tex}                                  % put in the main tex file


%% ~~~~~~ Necessary Packages
\usepackage[numbers]{natbib}                            % `numbers' option provides compact numerical references in the text. 
\usepackage{booktabs}                                   % for nice tables
\usepackage{multirow}                                   % for table cell taking multiple rows
\usepackage{makecell}                                   % for table cell with texts in multiple rows
\usepackage{tablefootnote}                              % for footnote in table
\usepackage[symbol]{footmisc}                           % for footnote symbol
\renewcommand{\thefootnote}{\arabic{footnote}}          % for footnote symbol: \alph, \arabic, \roman, \fnsymbol
\usepackage{amsmath,amssymb}                            % for \checkmark, etc.
\usepackage{xcolor}                                     % for colors
% \let\labelindent\relax                                % needed for legacy reasons in the IEEE template, together with `enumitem`
\usepackage{enumitem}                                   % for spaces in \itemize
\usepackage{subcaption}                                 % for subfigure
\usepackage{stfloats}                                   % for better place floats

\usepackage[bookmarks=true]{hyperref}     % Some templates forbid using this package. Be careful.
\usepackage{xspace}                       % To avoid missing space issues. Be careful: https://tex.stackexchange.com/a/86620

%%%% \newcommand{\commandname}[2]{{#1}content #2}
%%%% \def\commandname{content}


%% ~~~~~~ Math Commands
\newcommand{\field}[1]{\mathbb{#1}}
\newcommand{\R}{\field{R}}                              % real domain

\newcommand{\vct}[1]{\boldsymbol{#1}}                   % vector
\newcommand{\mat}[1]{\boldsymbol{#1}}                   % matrix
\newcommand{\tensor}[1]{\mathcal{#1}}                   % tensor
\newcommand{\T}{^{\top}}                                % transpose



%% ~~~~~ Format Commands
\newcommand{\para}[1]{\subsubsection{#1}}
%\newcommand{\para}[1]{\paragraph{#1}}
%\newcommand{\para}[1]{\noindent\textbf{#1}\ }


%% ~~~~~~ Edit Commands
\newcommand{\eat}[1]{}                                  % hide contents easily
\newcommand{\todo}[1]{{\color{red}#1}}                  % put `todo` comments
%\newcommand{\todo}[1]{#1}                              % hide `todo` comments
\newcommand{\TODO}[1]{{\textbf{\color{red}![#1]}}}      % put `TODO` comments
%\newcommand{\TODO}[1]{#1}                              % hide `TODO` comments

\newcommand{\zc}[1]{{\color{cyan}#1}}                   % add ZC's comments/changes
%\newcommand{\zc}[1]{#1}                                % hide ZC's comments/changes


%% ~~~~~ Rebuttal & Revision Commands
\newcommand{\question}[2]{\noindent \textbf{#1:} \textit{\color{blue}#2.}\ }  % Q&A for rebuttal
\newcommand{\dif}[1]{{\color{blue}#1}}                                        % switch to highlight contents, for revision
%\newcommand{\dif}[1]{#1}                                                     % switch to not highlight contents, for revision


%%%% For this project only
\usepackage{dirtree}
\newcommand{\ourslong}{You Only Look Once and One Thousand Times}
\newcommand{\ours}{YOLOv1001}
\newcommand{\oursx}{YOLOv1001\xspace}
\newcommand{\ug}[2]{{\color[rgb]{0,#1,0}\underline{#2}}}                                  % put in the main tex file


%% ~~~~~~ Necessary Packages
\usepackage[numbers]{natbib}                            % `numbers' option provides compact numerical references in the text. 
\usepackage{booktabs}                                   % for nice tables
\usepackage{multirow}                                   % for table cell taking multiple rows
\usepackage{makecell}                                   % for table cell with texts in multiple rows
\usepackage{tablefootnote}                              % for footnote in table
\usepackage[symbol]{footmisc}                           % for footnote symbol
\renewcommand{\thefootnote}{\arabic{footnote}}          % for footnote symbol: \alph, \arabic, \roman, \fnsymbol
\usepackage{amsmath,amssymb}                            % for \checkmark, etc.
\usepackage{xcolor}                                     % for colors
% \let\labelindent\relax                                % needed for legacy reasons in the IEEE template, together with `enumitem`
\usepackage{enumitem}                                   % for spaces in \itemize
\usepackage{subcaption}                                 % for subfigure
\usepackage{stfloats}                                   % for better place floats

\usepackage[bookmarks=true]{hyperref}     % Some templates forbid using this package. Be careful.
\usepackage{xspace}                       % To avoid missing space issues. Be careful: https://tex.stackexchange.com/a/86620

%%%% \newcommand{\commandname}[2]{{#1}content #2}
%%%% \def\commandname{content}


%% ~~~~~~ Math Commands
\newcommand{\field}[1]{\mathbb{#1}}
\newcommand{\R}{\field{R}}                              % real domain

\newcommand{\vct}[1]{\boldsymbol{#1}}                   % vector
\newcommand{\mat}[1]{\boldsymbol{#1}}                   % matrix
\newcommand{\tensor}[1]{\mathcal{#1}}                   % tensor
\newcommand{\T}{^{\top}}                                % transpose



%% ~~~~~ Format Commands
\newcommand{\para}[1]{\subsubsection{#1}}
%\newcommand{\para}[1]{\paragraph{#1}}
%\newcommand{\para}[1]{\noindent\textbf{#1}\ }


%% ~~~~~~ Edit Commands
\newcommand{\eat}[1]{}                                  % hide contents easily
\newcommand{\todo}[1]{{\color{red}#1}}                  % put `todo` comments
%\newcommand{\todo}[1]{#1}                              % hide `todo` comments
\newcommand{\TODO}[1]{{\textbf{\color{red}![#1]}}}      % put `TODO` comments
%\newcommand{\TODO}[1]{#1}                              % hide `TODO` comments

\newcommand{\zc}[1]{{\color{cyan}#1}}                   % add ZC's comments/changes
%\newcommand{\zc}[1]{#1}                                % hide ZC's comments/changes


%% ~~~~~ Rebuttal & Revision Commands
\newcommand{\question}[2]{\noindent \textbf{#1:} \textit{\color{blue}#2.}\ }  % Q&A for rebuttal
\newcommand{\dif}[1]{{\color{blue}#1}}                                        % switch to highlight contents, for revision
%\newcommand{\dif}[1]{#1}                                                     % switch to not highlight contents, for revision


%%%% For this project only
\usepackage{dirtree}
\newcommand{\ourslong}{You Only Look Once and One Thousand Times}
\newcommand{\ours}{YOLOv1001}
\newcommand{\oursx}{YOLOv1001\xspace}
\newcommand{\ug}[2]{{\color[rgb]{0,#1,0}\underline{#2}}}                                  % put in the main tex file


%% ~~~~~~ Necessary Packages
\usepackage[numbers]{natbib}                            % `numbers' option provides compact numerical references in the text. 
\usepackage{booktabs}                                   % for nice tables
\usepackage{multirow}                                   % for table cell taking multiple rows
\usepackage{makecell}                                   % for table cell with texts in multiple rows
\usepackage{tablefootnote}                              % for footnote in table
\usepackage[symbol]{footmisc}                           % for footnote symbol
\renewcommand{\thefootnote}{\arabic{footnote}}          % for footnote symbol: \alph, \arabic, \roman, \fnsymbol
\usepackage{amsmath,amssymb}                            % for \checkmark, etc.
\usepackage{xcolor}                                     % for colors
% \let\labelindent\relax                                % needed for legacy reasons in the IEEE template, together with `enumitem`
\usepackage{enumitem}                                   % for spaces in \itemize
\usepackage{subcaption}                                 % for subfigure
\usepackage{stfloats}                                   % for better place floats

\usepackage[bookmarks=true]{hyperref}     % Some templates forbid using this package. Be careful.
\usepackage{xspace}                       % To avoid missing space issues. Be careful: https://tex.stackexchange.com/a/86620

%%%% \newcommand{\commandname}[2]{{#1}content #2}
%%%% \def\commandname{content}


%% ~~~~~~ Math Commands
\newcommand{\field}[1]{\mathbb{#1}}
\newcommand{\R}{\field{R}}                              % real domain

\newcommand{\vct}[1]{\boldsymbol{#1}}                   % vector
\newcommand{\mat}[1]{\boldsymbol{#1}}                   % matrix
\newcommand{\tensor}[1]{\mathcal{#1}}                   % tensor
\newcommand{\T}{^{\top}}                                % transpose



%% ~~~~~ Format Commands
\newcommand{\para}[1]{\subsubsection{#1}}
%\newcommand{\para}[1]{\paragraph{#1}}
%\newcommand{\para}[1]{\noindent\textbf{#1}\ }


%% ~~~~~~ Edit Commands
\newcommand{\eat}[1]{}                                  % hide contents easily
\newcommand{\todo}[1]{{\color{red}#1}}                  % put `todo` comments
%\newcommand{\todo}[1]{#1}                              % hide `todo` comments
\newcommand{\TODO}[1]{{\textbf{\color{red}![#1]}}}      % put `TODO` comments
%\newcommand{\TODO}[1]{#1}                              % hide `TODO` comments

\newcommand{\zc}[1]{{\color{cyan}#1}}                   % add ZC's comments/changes
%\newcommand{\zc}[1]{#1}                                % hide ZC's comments/changes


%% ~~~~~ Rebuttal & Revision Commands
\newcommand{\question}[2]{\noindent \textbf{#1:} \textit{\color{blue}#2.}\ }  % Q&A for rebuttal
\newcommand{\dif}[1]{{\color{blue}#1}}                                        % switch to highlight contents, for revision
%\newcommand{\dif}[1]{#1}                                                     % switch to not highlight contents, for revision


%%%% For this project only
\usepackage{dirtree}
\newcommand{\ourslong}{You Only Look Once and One Thousand Times}
\newcommand{\ours}{YOLOv1001}
\newcommand{\oursx}{YOLOv1001\xspace}
\newcommand{\ug}[2]{{\color[rgb]{0,#1,0}\underline{#2}}}                                  % put in the main tex file


%% ~~~~~~ Necessary Packages
\usepackage[numbers]{natbib}                            % `numbers' option provides compact numerical references in the text. 
\usepackage{booktabs}                                   % for nice tables
\usepackage{multirow}                                   % for table cell taking multiple rows
\usepackage{makecell}                                   % for table cell with texts in multiple rows
\usepackage{tablefootnote}                              % for footnote in table
\usepackage[symbol]{footmisc}                           % for footnote symbol
\renewcommand{\thefootnote}{\arabic{footnote}}          % for footnote symbol: \alph, \arabic, \roman, \fnsymbol
\usepackage{amsmath,amssymb}                            % for \checkmark, etc.
\usepackage{xcolor}                                     % for colors
% \let\labelindent\relax                                % needed for legacy reasons in the IEEE template, together with `enumitem`
\usepackage{enumitem}                                   % for spaces in \itemize
\usepackage{subcaption}                                 % for subfigure
\usepackage{stfloats}                                   % for better place floats

\usepackage[bookmarks=true]{hyperref}     % Some templates forbid using this package. Be careful.
\usepackage{xspace}                       % To avoid missing space issues. Be careful: https://tex.stackexchange.com/a/86620

%%%% \newcommand{\commandname}[2]{{#1}content #2}
%%%% \def\commandname{content}


%% ~~~~~~ Math Commands
\newcommand{\field}[1]{\mathbb{#1}}
\newcommand{\R}{\field{R}}                              % real domain

\newcommand{\vct}[1]{\boldsymbol{#1}}                   % vector
\newcommand{\mat}[1]{\boldsymbol{#1}}                   % matrix
\newcommand{\tensor}[1]{\mathcal{#1}}                   % tensor
\newcommand{\T}{^{\top}}                                % transpose



%% ~~~~~ Format Commands
\newcommand{\para}[1]{\subsubsection{#1}}
%\newcommand{\para}[1]{\paragraph{#1}}
%\newcommand{\para}[1]{\noindent\textbf{#1}\ }


%% ~~~~~~ Edit Commands
\newcommand{\eat}[1]{}                                  % hide contents easily
\newcommand{\todo}[1]{{\color{red}#1}}                  % put `todo` comments
%\newcommand{\todo}[1]{#1}                              % hide `todo` comments
\newcommand{\TODO}[1]{{\textbf{\color{red}![#1]}}}      % put `TODO` comments
%\newcommand{\TODO}[1]{#1}                              % hide `TODO` comments

\newcommand{\zc}[1]{{\color{cyan}#1}}                   % add ZC's comments/changes
%\newcommand{\zc}[1]{#1}                                % hide ZC's comments/changes


%% ~~~~~ Rebuttal & Revision Commands
\newcommand{\question}[2]{\noindent \textbf{#1:} \textit{\color{blue}#2.}\ }  % Q&A for rebuttal
\newcommand{\dif}[1]{{\color{blue}#1}}                                        % switch to highlight contents, for revision
%\newcommand{\dif}[1]{#1}                                                     % switch to not highlight contents, for revision


%%%% For this project only
\usepackage{dirtree}
\newcommand{\ourslong}{You Only Look Once and One Thousand Times}
\newcommand{\ours}{YOLOv1001}
\newcommand{\oursx}{YOLOv1001\xspace}
\newcommand{\ug}[2]{{\color[rgb]{0,#1,0}\underline{#2}}}